%======================================================================
\chapter{Introduction}\label{intro}
%======================================================================
% look for citations about library usage
Software libraries are mainly collections of code components designed to be reused
across projects at different levels. They help developers save time and effort by
eliminating the need to rewrite code, and their functionalities can be directly integrated
into software applications.
The use of libraries developed by others is ubiquitous in modern
software development~\cite{huang22:_charac_java,wang20:_java}.
In recent years, the number of libraries available in different software ecosystems—such
as npm for JavaScript~\cite{ecmascript}, Maven Central for Java~\cite{java}, and PyPI for Python~\cite{python}—has grown significantly. For example, PyPI, the Python
software ecosystem, has exhibited a 47\% compound annual growth rate in active packages~\cite{Bommarito2019}.

A wide range of contributors—including: (i) Individual developers, (ii) Open-source communities, (iii) Companies
(such as Google, Facebook, and Microsoft) and (iv) Academic and Research
institutions~\cite{Lakhani2000OSS}—develop, maintain, and upgrade software libraries.
These contributors create libraries to meet specific needs and continue refining them over time.

The dependence on software libraries has increased over time because of the increase in open source software
{OSS} movement. Therefore, software development has become a
highly modular activity and has a huge reliance on third-party libraries~\cite{decan2019empirical, zerouali2019formal, cox2019surviving}.
However, libraries developed by others are also updated by others, on schedules that are not controlled by
the client developers.

% Especially when one is developing software that is exposed to the Internet, one
% has a responsibility to incorporate security updates for the
% libraries that one is using as a client~\cite{wu23:_under_threat_upstr_vulner_downs}, or else risk vulnerabilities
% being exposed in one's software~\cite{haryono22:_autom_ident_librar_vulner_data,zhan21:_atvhun,alfadel23:_empir_python}. 
% The obligation to update libraries is a form of technical debt that accrues automatically with the passage of time.

When a client develops software exposed to the internet, they take on a moral responsibility for ensuring that the
application integrates security updates for all utilized software libraries~\cite{wu23:_under_threat_upstr_vulner_downs}.
Failing to apply the latest updates can leave the application vulnerable to well-known security threats
such as Log4Shell (Log4j), Heartbleed (OpenSSL), and similar critical vulnerabilities~\cite{haryono22:_autom_ident_librar_vulner_data,zhan21:_atvhun,alfadel23:_empir_python}.
As such, clients must regularly update third-party libraries, since neglecting updates constitutes a
form of technical debt that accumulates automatically over time.



% Further, as these libraries are often
% developed and updated by
% professionals other than the software developers who integrate them into their own
% applications; while libraries allow developers to add functionality without building it
% from scratch, updates may introduce behavioural changes that can lead to unexpected
% outputs, potentially affecting the performance and results of the software~\cite{huang22:_charac_java,wang20:_java}.

% should we cite our Onward 2020 paper here?
% However, upgrading libraries is not painless~\cite{elizalde18:_towar_smoot_librar_migrat,derr17:_keep,dann23:_upcy}: new
% versions of libraries may include  breaking \gls{api} changes~\cite{dietrich14:_broken}, requiring developers to verify that their own client
% code continues working with the new library versions. This is
% inconvenient at best and can require nontrivial amounts of software development at worst,
% often without the reward of useful new features for the client software---reacting to upgrades
% just allows the client software to continue working, in a hopefully less-vulnerable
% state. 

However, upgrading a library to the latest version is not painless~\cite{elizalde18:_towar_smoot_librar_migrat,derr17:_keep,dann23:_upcy}:
new versions of libraries can include breaking \gls{api} changes~\cite{dietrich14:_broken}. In the case
of method signature-related changes, the library developer may have modified method signatures,
retracted previously available methods, or made other interface-level changes that are no longer
compatible with the client code. Worse yet, non-signature related breaking changes may trigger behavioural
changes in client application.
A \gls{bbc} caused by a non-method signature change can result from modified function logic or
a newly added unchecked exception in the function.
Clients can often catch these breaking changes with the help of test cases, provided they have created
tests to verify the behaviour of their application, especially as it relates to its
use of the library.
Upgrading the library can be inconvenient at best and may require nontrivial changes in the client
code at worst. Keeping libraries upgraded ensures that the client code
continues to work properly, and hopefully in a less-vulnerable manner.

Compilers and simple static
checkers (including japicmp\footnote{https://github.com/siom79/japicmp} and
Revapi\footnote{https://revapi.org/revapi-site/main/index.html} for Java as well
as \cite{brito18:_apidif,foo18:_effic_static_check_librar_updat})
can verify the absence of syntactic breaking changes in libraries. The situation
is worse for semantic/behavioural breaking changes:
there do not exist techniques for reliably detecting such changes. Of
course, in its full generality, the problem is undecidable, though
breaking change detection could be estimated using static and dynamic program analysis
techniques. One such technique has been implemented by CompCheck~\cite{CompCheck}.
The tool utilizes pre-written test cases present in the client code and runs them
while updating the library versions to find behavioural breaking changes (\gls{bbc}). Then,
the tool uses the pattern in which the \gls{api} was used to perform pattern matching
and identify more clients that may have a \gls{bbc}.

In this work, we contribute a novel way to detect one type of \gls{bbc} in a library.
Our approach enables client developers to inspect relevant changes to the set of
unchecked exceptions that a Java library may throw—particularly by the \gls{api}s
actually used by specific client code. A new unchecked exception thrown by a library
constitutes a \gls{bbc}; uncaught exceptions can cause the client code to crash or
exhibit unexpected behaviour. First, we collect all the \gls{api}s used by the client
and analyze them for any newly added unchecked exception(s). We perform this analysis
using static analysis tools and libraries such as SootUp~\cite{Karakaya24:_SootUp},
Soot~\cite{vallee2010soot}, and FlowDroid~\cite{Arzt14:_flowdroid}. Then, we conduct
a taint analysis to ensure that the unchecked exception is actually triggerable by
the client.

% Although developers tend to ignore even checked
% exceptions~\cite{nakshatri16:_analy_java}, we contend that incrementally informing
% developers only about relevant newly-added exceptions is likely to be more tractable, consistent with the
% design principles of Google's Tricorder tool~\cite{sadowski15:_tricor}.
% Thus, we leverage taint analysis
% to reduce the number of irrelevant reports that we report to client developers.
% We aim to show only changed library APIs that may realistically throw new exceptions
% in updated versions of client code, minimizing the number of false positives~\cite{pashchenko20:_vuln4,pashchenko18:_vulner}.
% We hope that our reports enable client developers to better understand how new exceptions affect their own code.

Developers often ignore checked exceptions~\cite{nakshatri16:_analy_java}, but we
contend that informing developers about relevant newly added exceptions as they upgrade
the library is likely to be more tractable, consistent with the design principles
of Google’s Tricorder tool~\cite{sadowski15:_tricor}. Therefore, we utilize taint
analysis to reduce the number of false positive reports sent to client developers.
By doing so, the number of reported \gls{api}s shown to the client includes only
those that may realistically throw new exceptions in updated versions of the client
code, minimizing false positives~\cite{pashchenko20:_vuln4}.
We hope that our reports help client developers better understand how new exceptions
affect their own code.

We explore the following research questions:

\noindent
{\bf RQ1.} How often do published changes to Java libraries throw new unchecked exceptions in methods,
and under what circumstances do such exceptions occur (e.g. major/minor/patch versions)?

\noindent
{\bf RQ2.} Do library clients, in practice, call methods with new added exceptions, and is it possible for the clients to trigger these exceptions? Is it possible to write client test cases that trigger the exceptions?

In our corpus of 69 distinct libraries, we found 24 libraries with newly-added exceptions, including exceptions that are added in non-major releases.
We then investigated 99 client-library pairs to explore the prevalence of potentially breaking behavioural changes.
We found that new potentially client-relevant unchecked exceptions occured in 8 of the 69 libraries, and that clients called methods reaching these exceptions at 49 client callsites.
This shows that client applications do in fact call library methods that throw these new exceptions.
Furthermore, we demonstrated that it is possible to trigger these exceptions by writing test cases using methods from the client.

The contributions of this work are as follows:

\begin{itemize}
    \item We implement the \textit{UnCheckGuard} static analysis tool, which traverses bytecode to find newly-added exceptions and filters them using taint analysis, to report relevant newly added unchecked exceptions.
    \item We conduct an empirical study of libraries to detect potential behavioural breaking changes in libraries caused by newly added unchecked exceptions.
    \item We evaluate 99 client-library pairs from the DUETS dataset~\cite{durieux21:_duets} using \textit{UnCheckGuard}, identifying 49 call sites where libraries' newly added unchecked exceptions could cause behavioural breaking changes in clients, and write test cases showing that the exceptions can be triggered from client code.
\end{itemize}

\noindent \textbf{Data Availability Statement.}

\indent We have made our tool and dataset available at \url{https://doi.org/10.5281/zenodo.16788650}