%======================================================================
\chapter{Background}\label{background}
%======================================================================

In this chapter we define core concepts that underpin our approach to detecting
\gls{bbc}s caused by newly added exceptions.

\textbf{Exception Handling}. Exception handling is a tool that allows developers to recover from
exceptional or error conditions that may disrupt the intended flow of an application~\cite{Suman2016exception}.
In particular, Java supports the use of \texttt{try-catch} blocks to handle exceptions.
The following is an example of exception handling in Java using a \texttt{try-catch}
block:

\begin{lstlisting}[language=java]
public class ExceptionHandlingExample {
    public static void main(String[] args) {
        try {
            int result = Math.floorDiv(10,0);
            System.out.println("Result: " + result);
        } catch (ArithmeticException e) {
            System.out.println("Something went wrong: " + e.getMessage());
        }
    }
}
\end{lstlisting}

In the above example, dividing 10 by 0 using \texttt{Math.floorDiv} throws an
\texttt{\seqsplit{ArithmeticException}}, as specified in the official documentation of the
\texttt{Math} class in the Java Standard Library.\footnote{\url{https://docs.oracle.com/javase/8/docs/api/java/lang/Math.html}}
The \texttt{try-catch} blocks catch and handle the \texttt{ArithmeticException}
to prevent the program from crashing.



\textbf{Java Exceptions}. Java defines two different types of exceptions:
\begin{itemize}
    \item \textbf{Checked Exception}: appears in the throwing method's signature. The client must either catch the exception or declare it to be
    thrown~\cite{Sousa2020evolution}. The compiler checks this type of exception, as do several tools
    such as japicmp\footnote{\url{https://github.com/siom79/japicmp}}. The Java Language Specification~\cite{Gosling2021java}
    formally defines the semantics of checked exceptions and compiler behaviour. Following is an example
    of a checked exception:
    \begin{lstlisting}[language=java]
public static void readFile() throws FileNotFoundException {
    FileReader file = new FileReader("data.txt");
    System.out.println("Reading file...");
    file.close();
}
    \end{lstlisting}
    In this example, the public method \texttt{readFile()} declares that it throws an \texttt{\seqsplit{FileNotFoundException}},
    which is a checked exception in Java. If the method cannot read the file \texttt{data.txt}
    or if the file does not exist, the \texttt{FileReader} constructor\footnote{\url{https://docs.oracle.com/javase/8/docs/api/java/io/FileReader.html}} throws a \texttt{FileNotFoundException}.
    The developer must handle this exception either by using a \texttt{try-catch} block or by declaring
    it in the method signature with the \texttt{throws} keyword.

    \item \textbf{Unchecked Exception}: do not appear in the throwing method's
    signature, and includes subclasses of RuntimeException or Error. This type
    of exception does not appear in the method's signature~\cite{Asaduzzaman2017}. As a result, the
    compiler does not check it, and tools such as japicmp do not detect it. These exceptions
    can cause unexpected runtime failures when the client does not handle them correctly~\cite{Padua2017}.
    This type of exceptions often gets overlooked by client developers particularly during testing,
    especially when they are introduced through library upgrades. Client developers often overlook
    this type of exception during testing, especially when library upgrades introduce them. The
    addition of unchecked exceptions to newer versions of libraries can contribute to \gls{bbc}s in
    the client applications. Following is an example of an unchecked exception:
    \begin{lstlisting}[language=java]
public class ThrowUncheckedExample {

    public static void main(String[] args) {
        boolean flag = false;
        checkFlag(flag);
    }

    public static void checkFlag(boolean flag) {
        if (!flag) {
            throw new IllegalArgumentException("Flag is false");
        }
        System.out.println("Flag is true");
    }
}
    \end{lstlisting}
    In this example, the method \texttt{checkFlag} explicitly throws an \texttt{\seqsplit{IllegalArgumentException}},
    which is an unchecked exception. Since it is a subclass of \texttt{\seqsplit{RuntimeException}}, the Java compiler
    does not require the \texttt{checkFlag} method to declare it using the \texttt{throws} keyword or
    to check whether the calling function handles it using a \texttt{try-catch} block.
\end{itemize}

\textbf{Why unchecked exceptions exist}. Java includes unchecked exceptions to represent programming errors that developers should fix rather than handle using a \texttt{try-catch} block. Unchecked exceptions usually indicate issues such as invalid arguments, logic bugs, or null references. Since these exceptions reflect flaws in the program's logic, Java does not require developers to declare them in a method’s signature. This design choice helps avoid forcing developers to catch exceptions they cannot reasonably recover from.

\textbf{Transitive Throwing.} Unchecked exceptions can propagate through multiple method calls without being declared. When a method throws an unchecked exception and does not catch it, the exception moves up the call chain until the program either catches it or crashes. We refer to this behavior as transitive throwing.

This becomes problematic when library methods introduce new unchecked exceptions. Even if the client does not directly call the modified method, the exception can still reach client code through intermediate methods. This silent propagation of transitive unchecked exceptions can potentially cause a \gls{bbc}.

\textbf{Breaking Changes}. We define a breaking change as a change in the library's \gls{api} that
causes the client code to either break or stop functioning the way it did prior to the library upgrade.
Breaking changes fall into two categories:
\begin{itemize}
    \item \textbf{Syntactic Breaking Changes}. This type of change usually occurs when the method
    signature changes. Library developers may change the method's signature by removing or updating
    the function name, modifying the input parameters, changing the checked exception(s) associated
    with the function, or altering the return type~\cite{jayasuriya24}. Static tools like japicmp
    can detect these types of method signature changes. The Java compiler checks for syntactic breaking
    changes during compilation but it only checks the code that is being recompiled. The Java compiler
    will not report any errors if the JAR of a library is replaced during runtime without recompiling
    (drop-in change). An example of a syntactic breaking change follows. We first present the
    code before changes:
    \begin{lstlisting}[language=java]
public class MyLibrary {
    public void greet() {
        System.out.println("Hello!");
    }
}
    \end{lstlisting}
    Now, we present the code after the changes, which introduces a syntactic breaking change:
    \begin{lstlisting}[language=java]
public class MyLibrary {
    public void greet(String name) {
        System.out.println("Hello, " + name + "!");
    }
}
    \end{lstlisting}
    In this example, the original version defines the method \texttt{greet()} with no parameters. In
    the updated version, the method \texttt{greet()} requires a parameter of type \texttt{String}. If
    the client application, which used the original version, tries to recompile the code against the
    updated version, the compiler raises a compilation error, indicating that no method
    \texttt{greet()} with zero arguments exists.

    \item \textbf{Behavioural Breaking Changes (\gls{bbc})}. In this type of change, the syntax remains
    the same, but the semantics change. Various reasons can cause such changes in semantics, including
    updates to the function logic, addition of a new unchecked exception, or modification of an existing
    unchecked exception (for example, changing an \texttt{IllegalArgumentException} into a
    \texttt{NullPointerException}). In general, the compiler does not detect these types of changes.
    They can cause the client's application to crash at runtime. The following is an example of a
    behavioural breaking change. We first present the original version of the method before any modifications:

    \begin{lstlisting}[language=java]
public class FeatureToggle {
    public boolean isFeatureEnabled() {
        return true;
    }
}
    \end{lstlisting}
Client:
\begin{lstlisting}
FeatureToggle ft = new FeatureToggle();
if (ft.isFeatureEnabled()) {
    System.out.println("New feature is active!");
}
\end{lstlisting}
The client prints ``{New feature is active!}'' when using the original version of the \texttt{isFeatureEnabled}
method.

Now, we present the updated version of the method:
\begin{lstlisting}[language=java]
public class FeatureToggle {
    public boolean isFeatureEnabled() {
        return false;  // Changed behaviour
    }
}
\end{lstlisting}
In this example, the original version of the \texttt{isFeatureEnabled} method always returns \texttt{true},
which causes the client to print ``{New feature is active!}''. After the update, the method's logic changes
to always return \texttt{false}. As a result, the client no longer prints the message. Although the method
signature remains unchanged, the internal behaviour differs. This change breaks the intended behaviour
of the client application.
\end{itemize}

\textbf{Semantic Versioning}. Software libraries generally follow semantic versioning \\(semver)~\cite{preston-werner23:_seman_version}, where the version number of the library indicates the level of change. Developers structure the numbers as "MAJOR.MINOR.PATCH":
\begin{itemize}
    \item \textbf{MAJOR} version number flags breaking changes in the library.
    \item \textbf{MINOR} version number indicates the introduction of new features while ensuring that everything from the previous version still works (backward compatibility).
    \item \textbf{PATCH} version number refers to bug fixes only.
\end{itemize}
However, in practice, developers often introduce breaking changes even in minor or patch versions~\cite{jayasuriya24:_under_apis}. This behaviour makes it especially important to create and use tools that check behavioural compatibility instead of relying solely on version numbers.

\textbf{Static Analysis.} Static analysis involves debugging the source code or bytecode
of a program without executing it. Developers use it to analyze potential errors and
security vulnerabilities, and they can also apply it to support compiler optimizations.
It is particularly useful when dynamic test cases are not available,
incomplete, or insufficient, and allows developers to get prior information about possible
errors and issues that may occur during actual execution~\cite{Rahaman2023}.

Static analysis is particularly useful in analyzing large programs, especially those that
incorporate multiple libraries. By providing early feedback to developers,
static analysis reduces the likelihood of errors, thereby helping to maintain the reliability
and performance of the software. Despite its benefits, static analysis also presents several
drawbacks: (i) it often produces false positives, especially in complex codebases with dynamic
behaviour such as Java reflection, dynamic class loading, or external dependencies; and (ii)
since it analyzes code without executing it, it cannot always determine the actual execution
path, which limits its precision.

\textbf{Dynamic Analysis.} Dynamic analysis runs the actual program to observe its
behaviour during execution. It can potentially detect runtime issue caught
during execution, but it relies on test coverage and can miss particular cases when a test
case is not available for that case~\cite{Kuliamin2024}. Dynamic analysis is helpful in analysing software
behaviour. It does so by analysing program operations during execution.

Further, dynamic analysis has following advantages: (i) it minimizes false positives by reporting only those faults that actually
occur during execution, (ii) It proves particularly useful in detecting security-related issues,
including buffer overflows, improper input handling, and unauthorized access, provided that a
suitable test case is already present. Despite these advantages, dynamic analysis has certain limitations
also and same are as follows: (i) Its effectiveness mainly depends on the test cases used;
(ii) part of code that is not executed during testing will not be analyzed for potential
errors, and (iii) It is also sometimes hard to actually run a large software system. Static
analysis is sometimes more viable for these systems, even if it has to approximate the behaviour.

\textbf{Taint Analysis}. Taint analysis is a program analysis technique which can be implemented statically~\cite{myers99:_jflow} or dynamically~\cite{newsome05:_dynam}.
It relies on declarations of sources (for example, client input) and sinks (for example, critical
operations or exceptions). Given these inputs, the analysis tracks whether the sources can reach the sinks.

The following example demonstrates how taint flows from a source to a sink:

\begin{lstlisting}[language=java]
public class FlowDroidExampleCode {
  public static int source() { return 1337; }

  public void exampleTaintAnalysis() {
    int temp = source(); int[] arr = new int[2];
    arr[0] = temp; arr[1] = 19;
    if (arr[0] == 1337) {
      throw new RuntimeException("hello"); }
    }
}
\end{lstlisting}

In this example, the method \texttt{source()} acts as the taint source. The statement \texttt{throw new RuntimeException("hello")} is the sink. The tainted value flows into the array \texttt{arr}, and later influences the conditional that triggers the exception. Although the exception is hardcoded, the fact that its execution depends on a tainted value makes this a valid taint flow from the source to the sink.

We apply taint analysis to detect whether newly added exceptions in a library are reachable from client-supplied values. This helps us detect behavioural breaking changes where a newly added unchecked exception is only triggered under specific conditions influenced by the client.

\textbf{Call Graph Analysis}. In static analysis, call graphs identify the function calls that the
program will make. Nodes represent methods, and edges represent calls from one method to another.
This technique plays a fundamental role in data flow analysis, control flow analysis,
dead code elimination, and taint tracking~\cite{Keshani2024}.
For example, consider the following Java program:
\begin{lstlisting}[language=java]
public class Example {
    public static void main(String[] args) {
        int n = 10;
        int d = 0;
        doWork(n,d);
}

public static void doWork(int n, int d) {
    if ( d == 0){
        throw new IllegalArgumentException("denominator cannot be zero");
    }
    calculate(n,d);
}

public static void calculate(int n, int d) {
    int result = Math.floorDiv(n,d);
}
\end{lstlisting}

\begin{figure}[h]
\centering
\begin{tikzpicture}[
    node distance=1cm and 2.5cm,
    every node/.style={draw, rounded corners, align=center, minimum height=1cm},
    exception/.style={draw=red, thick, rounded corners, align=center, minimum height=1cm, fill=red!10},
    libcall/.style={draw=blue, thick, rounded corners, align=center, minimum height=1cm, fill=blue!10},
    arrow/.style={->, thick}
]

% Nodes
\node (main) {main()};
\node (doWork) [below=of main] {doWork(int, int)};
\node (calculate) [below=of doWork] {calculate(int, int)};
\node[exception] (iae) [right=of doWork] {throw \\ IllegalArgumentException};
\node[libcall] (floorDiv) [below=of calculate] {Math.floorDiv(n, d)};

% Edges
\draw[arrow] (main) -- (doWork);
\draw[arrow] (doWork) -- (calculate);
\draw[arrow] (doWork) -- (iae);
\draw[arrow] (calculate) -- (floorDiv);

\end{tikzpicture}
\caption{Call graph for the \texttt{Example} program showing method calls and exception sources.}
\label{fig:callgraph-example}
\end{figure}

Figure~\ref{fig:callgraph-example} illustrates the call graph for the given Java program. The graph shows the sequence of method calls starting from the \texttt{main()} method, through \texttt{doWork()}, and into \texttt{calculate()}. Inside \texttt{doWork()}, a critical operations occur: a user-defined unchecked exception (\texttt{IllegalArgumentException}) is explicitly thrown when the denominator is zero.

This call graph is useful in tracing the flow of potential exceptions-both user-defined and library-sourced-through the client code. If a library version introduces a new unchecked exception at any point in this call chain (e.g., inside \texttt{Math.floorDiv()}), the static call graph can help identify all the client-side methods that may now be affected by that change. This is essential for detecting \glspl{bbc}.

\textbf{DUETS Dataset}. The DUETS dataset~\cite{durieux21:_duets} provides a list of real-world Java
client-library pairs. Each client in the DUETS dataset has over 5 stars on GitHub. For our evaluation,
we selected 1011 clients from DUETS in a systematic manner, rather than using the
entire dataset of 147,991 clients. The DUETS dataset provides a list of Java-based clients that are
more actively maintained and show better community engagement, as all the projects have over 5 stars.
It represents real-world library usage in Java-based projects.

\textbf{SootUp}. SootUp~\cite{Karakaya24:_SootUp} is a respin of the Soot~\cite{vallee2010soot} framework that supports static analysis of Java bytecode.

SootUp transforms JVM bytecode into the intermediate representation Jimple, which simplifies analysis by converting low-level bytecode instructions into a higher-level format that makes method bodies, variable assignments, exception handling blocks, and method invocations accesible. SootUp also provides call graph generation with various algorithms and precision levels. When a library method throws a new unchecked exception, we use SootUp to determine whether client methods transitively call that library method by traversing the (Class Hierarchy Analysis) call graph. We also use the Jimple intermediate representation to inspect methods that may throw an exception, by examining throw statements and method calls within their bodies.

\textbf{FlowDroid}. FlowDroid~\cite{Arzt14:_flowdroid} is a static taint analysis framework. It tracks data flow from declared sources to sinks within the application's code. It is built on top of the Soot~\cite{vallee2010soot} static analysis framework. FlowDroid models the complete Android lifecycle and callback structure---irrelevant for our purposes---but, relevant to us, enables flow-sensitive, field-sensitive, context-sensitive, and object-sensitive analysis of both Android and normal Java Virtual Machine programs.

It checks whether data from a source will taint a sink by computing possible paths along which the data can flow. In our tool, we use taint analysis to check the approximate reachability of newly added unchecked exceptions from client code.
