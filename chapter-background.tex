%======================================================================
\chapter{Background}\label{background}
%======================================================================

In this chapter we define core concepts that underpin our approach to detecting
\gls{bbc} caused by newly added exceptions.

\textbf{Exception Handling}. Exception handling is a powerful tool that allows developers to recover from
exceptional or error conditions that may disrupt the intended flow of an application~\cite{Suman2016exception}.
In particular, Java supports the use of \texttt{try-catch} blocks to handle exceptions.
The following is an example of exception handling in Java using a \texttt{try-catch}
block:

\begin{lstlisting}[language=java]
public class ExceptionHandlingExample {
    public static void main(String[] args) {
        try {
            int result = Math.floorDiv(10,0);
            System.out.println("Result: " + result);
        } catch (ArithmeticException e) {
            System.out.println("Something went wrong: " + e.getMessage());
        }
    }
}
\end{lstlisting}

In the above example, dividing 10 by 0 using \texttt{Math.floorDiv} throws an
\texttt{ArithmeticException}, as specified in the official documentation of the
\texttt{Math} class in the Java Standard Library.\footnote{\url{https://docs.oracle.com/javase/8/docs/api/java/lang/Math.html}}
The \texttt{try-catch} blocks catch and handle the \texttt{ArithmeticException}
to prevent the program from crashing.



\textbf{Java Exceptions}. Java defines two different types of exceptions:
\begin{itemize}
    \item \textbf{Checked Exception}, appears in the method's signature. When a client uses an
    \gls{api} with a checked exception, the client either catches the exception or declares it to be
    thrown~\cite{Sousa2020evolution}. The compiler checks this type of exception, as do several tools
    such as japicmp\footnote{https://github.com/siom79/japicmp}. The Java Language Specification~\cite{Gosling2021java}
    formally defines the semantics of checked exceptions and compiler behavior. Following is an example
    of a checked exception:
    \begin{lstlisting}[language=java]
public static void readFile() throws FileNotFoundException {
    FileReader file = new FileReader("data.txt");
    System.out.println("Reading file...");
    file.close();
}
    \end{lstlisting}
    In this example, the public method \texttt{readFile()} declares that it throws an \texttt{FileNot\\FoundException},
    it is a checked exception in Java. If the method cannot read the file \texttt{data.txt}
    or if the file does not exist, the \texttt{FileReader} constructor\footnote{\url{https://docs.oracle.com/javase/8/docs/api/java/io/FileReader.html}} throws an \texttt{FileNotFoundException}.
    The developer must handle this exception either by using a \texttt{try-catch} block or by declaring
    it in the method signature with the \texttt{throws} keyword.

    \item \textbf{Unchecked Exception}, includes subclasses of RuntimeException or Error. This type
    of exception does not appear in the method's signature~\cite{Asaduzzaman2017}. As a result, the
    compiler does not check it, and static tools such as japicmp do not detect it. These exceptions
    can cause unexpected runtime failures when the client does not handle them correctly~\cite{Padua2017}.
    This type of exceptions often gets overlooked by Client developers particularly during testing,
    especially when they are introduced through library upgrades. Client developers often overlook
    this type of exception during testing, especially when library upgrades introduce them. The
    addition of unchecked exceptions to newer versions of libraries can contribute to \gls{bbc}s in
    the client applications. Following is an example of an unchecked exception:
    \begin{lstlisting}[language=java]
public class ThrowUncheckedExample {

    public static void main(String[] args) {
        boolean flag = false
        checkFlag(flag);
    }

    public static void checkFlag(boolean flag) {
        if (!flag) {
            throw new IllegalArgumentException("Flag is false");
        }
        System.out.println("Flag is true");
    }
}
    \end{lstlisting}
    In this example, the method \texttt{checkFlag} explicitly throws an \texttt{IllegalArgument\\Exception},
    which is an unchecked exception. Since it is a subclass of \texttt{Runtime\\Exception}, the Java compiler
    does not require the \texttt{checkFlag} method to declare it using the \texttt{throws} keyword or
    to check whether the calling function handles it using a \texttt{try-catch} block.
\end{itemize}

\textbf{Breaking Changes}. We define a breaking change as a change in the library's \gls{api} that
causes the client code to either break or stop functioning the way it did prior to the library upgrade.
Breaking changes fall into two categories:
\begin{itemize}
    \item \textbf{Syntactic breaking changes}. This type of change usually occurs when the method's
    signature changes. Library developers may change the method's signature by removing or updating
    the function name, modifying the input parameters, changing the checked exception(s) associated
    with the function, or altering the return type~\cite{jayasuriya24}. Static tools like japicmp
    can detect these types of method signature changes. The Java compiler checks for syntactic breaking
    changes during compilation but it only checks the code that is being recompiled. The Java compiler
    will not report any errors if the JAR of a library is replaced during runtime without recompiling
    (drop-in change). Following is an exmaple of syntactic breaking change, we first present the code
    before changes:
    \begin{lstlisting}[language=java]
public class MyLibrary {
    public void greet() {
        System.out.println("Hello!");
    }
}
    \end{lstlisting}
    Now, we present the code after the changes, which introduces a syntactic breaking change:
    \begin{lstlisting}[language=java]
public class MyLibrary {
    public void greet(String name) {
        System.out.println("Hello, " + name + "!");
    }
}
    \end{lstlisting}
    In this example, the original version defines the method \texttt{greet()} with no parameters. In
    the updated version, the method \texttt{greet()} requires a parameter of type \texttt{String}. If
    the client application, which used the original version, tries to recompile the code against the
    updated version, the compiler raises a \texttt{compilation error}, indicating that no method
    \texttt{greet()} with zero arguments exists.

    \item \textbf{Behavioural breaking changes (\gls{bbc})}. In this type of change, the syntax remains
    the same, but the semantics change. Various reasons can cause such changes in semantics, including
    updates to the function logic, addition of a new unchecked exception, or modification of an existing
    unchecked exception (for example, changing an \texttt{IllegalArgumentException} into a
    \texttt{NullPointerException}). The compiler does not detect these types of changes.
    They can cause the client's application to crash at runtime. The following is an example of a
    behavioural breaking change. We first present the original version of the method before any modifications:

    \begin{lstlisting}[language=java]
public class FeatureToggle {
    public boolean isFeatureEnabled() {
        return true;
    }
}
    \end{lstlisting}
Client:
\begin{lstlisting}
FeatureToggle ft = new FeatureToggle();
if (ft.isFeatureEnabled()) {
    System.out.println("New feature is active!");
}
\end{lstlisting}
The client prints ``{New feature is active!}'' when using the original version of the \texttt{isFeatureEnabled}
method.

Now, we present the updated version of the method:
\begin{lstlisting}[language=java]
public class FeatureToggle {
    public boolean isFeatureEnabled() {
        return false;  // Changed behavior
    }
}
\end{lstlisting}
In this example, the original version of the \texttt{isFeatureEnabled} method always returns \texttt{true},
which causes the client to print ``{New feature is active!}''. After the update, the method's logic changes
to always return \texttt{false}. As a result, the client no longer prints the message. Although the method
signature remains unchanged, the internal behavior differs. This change breaks the intended behaviour
of the client application.
\end{itemize}

\textbf{Semantic Versioning}. Software libraries generally follow semantic versioning \\(semver)~\cite{preston-werner23:_seman_version}, where the version number of the library indicates the level of change. Developers structure the numbers as "MAJOR.MINOR.PATCH":
\begin{itemize}
    \item \textbf{MAJOR} version number flags breaking changes in the library.
    \item \textbf{MINOR} version number indicates the introduction of new features while ensuring that everything from the previous version still works (backward compatibility).
    \item \textbf{PATCH} version number refers to bug fixes only.
\end{itemize}
However, in practice, developers often introduce breaking changes even in minor or patch versions~\cite{jayasuriya24:_under_apis}. This behavior makes it especially important to create and use tools that check behavioural compatibility instead of relying solely on version numbers.

\textbf{Static Analysis.} Static analysis involves debugging the source code or bytecode
of a program without executing it. Developers use it to analyze potential errors and
security vulnerabilities, and they can also apply it to support compiler optimizations.
It is particularly useful when dynamic test cases are not available,
incomplete, or insufficient, and allows developers to get prior information about possible
errors and issues that may occur during actual execution~\cite{Rahaman2023}.

Static analysis is particularly useful in analyzing large programs, especially those that
incorporate multiple libraries. By providing early feedback to developers,
static analysis reduces the likelihood of errors, thereby helping to maintain the reliability
and performance of the software.

\textbf{Dynamic Analysis.} Dynamic analysis runs the actual program to observe its
behaviour during execution. It can potentially detect runtime issue caught
during execution, but it relies on test coverage and can miss particular cases when a test
case is not available for that case~\cite{Kuliamin2024}. Dynamic analysis is helpful in analysing software
behaviour. It does so by analysing program operations during execution.

Further, dynamic analysis has following advantages: (i) it minimizes false positives by reporting only those faults that actually
occur during execution, (ii) It proves particularly useful in detecting security-related issues,
including buffer overflows, improper input handling, and unauthorized access, provided that a
suitable test case is already present. Despite these advantages, dynamic analysis has certain limitations
also and same are as follows: (i) Its effectiveness mainly depends on the test cases used;
(ii) code string that is not executed during testing will not be analyzed for potential
errors, and (iii) it may also be less practical for analyzing large-size software or
applications~\cite{Somi2024}.

\textbf{Taint Analysis}. Taint analysis is a static or dynamic program analysis technique.
In taint analysis, sources (for example, client input) and sinks (for example, critical
operations or exceptions) are declared, and then it tracks whether the sources can reach the sinks.


\textbf{Call Graph Analysis}. In static analysis, call graphs identify the function calls that the
program will make. This technique plays a fundamental role in data flow analysis, control flow analysis,
dead code elimination, and taint tracking~\cite{Keshani2024}. A call graph is constructed without
executing the code, using class hierarchies, method signatures, and type information. CHA assumes
that any method that might be invoked based on the class hierarchy and method overrides can be called.

\textbf{DUETS Dataset}. The DUETS dataset~\cite{durieux21:_duets} provides a list of real-world Java
client-library pairs. Each client in the DUETS dataset has over 5 stars on GitHub. For our evaluation,
we selected a convenience sample from the first few hundred clients in DUETS, rather than using the
entire dataset of 147,991 clients. The DUETS dataset provides a list of Java-based clients that are
more actively maintained and show better community engagement, as all the projects have over 5 stars.
It represents real-world library usage in Java-based projects.