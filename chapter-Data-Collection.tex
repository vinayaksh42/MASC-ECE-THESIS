%======================================================================
\chapter{Data Collection}
%======================================================================
% This section describes the systematic approach we used to construct the dataset for our study on behavioural incompatibilities caused by newly added unchecked exceptions in upgraded Java libraries. 

This section highlights the approach we use to construct the dataset for our study on behavioural incompatibilities caused by newly added unchecked exceptions in upgraded Java libraries.

% Broadly, we require three sets of components: Java-based clients that depend on third-party libraries; the versions of the libraries declared as dependencies by these clients (``current'' versions); and the latest available versions of those same libraries (``latest'' versions). For each client-library pair, we also need to extract the set of unchecked exceptions thrown by the library methods actually used by the client.

The tool UnCheckGuard requires three components to perform its analysis: Java-based clients that depend on third-party libraries; the versions of the libraries these clients currently declare as dependencies (“current” versions); and the latest available versions of those same libraries (“latest” versions). For each client, the tool extracts the unchecked exceptions thrown by both the current and the latest versions of the library. It then formulates a list of methods that, when upgraded to the newer version, introduce a newly added unchecked exception. The tool focuses on extracting and storing this information only for the methods that the client application actually uses.

% To collect data that our UnCheckGuard tool will use, we carry out the following steps: identifying suitable Java clients; extracting their library dependencies; resolving both the current and latest versions of the libraries; analyzing exception behaviour in both versions; and recording all methods that introduce newly added unchecked exceptions. Using this data, UnCheckGuard can report client call sites that may be affected by behavioural breaking changes in a library upgrade.

To collect the data mentioned above, our tool UnCheckGuard follows these steps:
\begin{itemize}
    \item Identify suitable Java clients.
    \item Extract their library dependencies and resolve both the current and latest versions of the libraries.
    \item Extracting methods for analysis.
    \item Record all methods that introduce newly added unchecked exceptions.
\end{itemize}
With this data, UnCheckGuard reports the call sites present in the client application that may be affected by behavioural breaking changes during a library upgrade.

\section{Collecting Clients}

% To begin our analysis, we first collected suitable client projects. We used the DUETS dataset~\cite{durieux21:_duets}, which provides a curated list of Java-based clients hosted on GitHub, each with at least five stars. DUETS also pairs libraries with the clients, but we ignore the DUETS library declarations and instead consider all of the libraries declared as dependencies by each client.

To start our analysis, we created a \texttt{Python} script to collect suitable client projects. For this process, we used the DUETS dataset~\cite{durieux21:_duets}, which provides a curated list of Java-based clients hosted on GitHub. We selected this list because all Java projects in it have at least 5 stars on GitHub. We chose projects with a minimum of 5 stars as this indicates a baseline level of community engagement; better community engagement generally reflects a more actively maintained project. This approach helps us collect clients that represent real-world library usage in Java-based projects.

% We took a convenience sample of the first few hundred clients from DUETS rather than the entire DUETS dataset of 147,991 clients. We attempted to download each client repository and discarded any client that failed to download.

% Next, we checked whether the project included a \texttt{pom.xml} file, which indicates that it is a Maven-based project. This step was essential, as our analysis depends on running Maven commands. We compiled each client to produce a JAR file and kept only those clients that compiled successfully for further analysis. After this process, we were left with 36 Maven-based clients that we had successfully compiled.

We ran our \texttt{Python} script on the first few hundred clients from the DUETS dataset, rather than on the entire dataset of 147,991 clients. We attempted to download each client repository and discarded any client that failed to download. 

Next, we checked whether the project included a \texttt{pom.xml} file; the presence of this file indicates that the project is Maven-based. We discarded any project that was not Maven-based, as our analysis depends on running Maven commands. The \texttt{Python} script then checked the Java version specified in the \texttt{pom.xml} file, and based on that version, we ran the Maven commands using the corresponding Java environment. We compiled each client to produce a JAR file and retained only those clients that compiled successfully for further analysis. After completing this filtering process, we were left with 36 Maven-based clients that we had successfully compiled.


\section{Library Version Resolution}

% Our tool relies on analyzing both the version of the library currently used by the client and the latest available version (as stored in the Maven Central Repository). To collect the current version, we run the Maven command \texttt{mvn dependency:copy-dependencies}, which downloads all the dependencies declared in the client's build configuration.

UnCheckGuard depends on analyzing both the current version of the library used by the client and the latest available version of the library, as stored in the Maven Central Repository. To collect both versions, we run a set of Maven commands. To retrieve the current version of the library, we run the following set of Maven commands:

\begin{lstlisting}[language=bash]
mvn clean package -DskipTests -fn
mvn dependency:copy-dependencies
\end{lstlisting}

This command allows us to store all the dependencies currently used by the client.

To obtain the latest version of the libraries, we run the following Maven command:
\begin{lstlisting}[language=bash]
mvn org.codehaus.mojo:versions-maven-plugin:2.18.0:use-latest-versions
mvn clean package -DskipTests -fn
mvn dependency:copy-dependencies
\end{lstlisting}
This command updates the \texttt{pom.xml} file with the most recent versions of all declared dependencies. We then re-run \texttt{mvn dependency:copy-dependencies} to download the updated set of libraries.

This process fetches both the current and the latest versions of each library used by the client, enabling us to perform a comparative analysis of behavioural changes across library versions. Out of the 36 clients that we collected, we found 26 clients with different current and latest versions for some library. These clients collectively used 17 libraries with different versions, with 22 current versions used by some client, updated to 17 latest versions.

\section{Method Extraction}

% We analyze the JAR file of the client application using SootUp~\cite{Karakaya24:_sootup} to extract all external method invocations performed by the client (by external, we mean to methods outside the client). Based on this set of method calls, we analyze both the version of the library that the client currently uses as well as its latest available version.

UnCheckGuard then begins analyzing the clients we collected in the first step. It uses SootUp~\cite{Karakaya24:_sootup} to extract all external method invocations performed by the client (by external, we refer to methods outside the client). We store a list of all external method invocations made by the client in a \texttt{JSON} file. This file is later used to compare against the different methods within the libraries that the client uses.

To support this comparison, we extract the list of methods present in the current version of the library. This allows us to match each external method call made by the client to its corresponding definition in the library. Using this information, we generate a JSON file that contains only the subset of library methods actually used by the client. This filtered list reduces the work necessary in subsequent steps.


\section{Comparing Library Versions}

UnCheckGuard then analyzes the methods used by the client in both the current and the latest available versions of the library. The tool scans for exceptions and performs taint analysis to verify the reachability from the client to the exceptions present in the method. After collecting the unchecked exceptions from both versions of the library, we store the list of exceptions present in each method in a \texttt{JSON} file. We use the following format to store this information:
\begin{lstlisting}[language=python]
[
    ...
    {"org.apache.http.HttpHost": [{
        "methodSignature": "<org.apache.http.HttpHost: void <init>(java.lang.String,int)>",
        "unchecked_exceptions": [
            "java.lang.IllegalArgumentException <org.apache.http.util.Args: java.lang.CharSequence containsNoBlanks(java.lang.CharSequence,java.lang.String)>",
            "java.lang.IllegalArgumentException <org.apache.http.util.Args: java.lang.CharSequence containsNoBlanks(java.lang.CharSequence,java.lang.String)>"
        ]
    }]},
    ...
]
\end{lstlisting}

This \texttt{JSON} file contains a list of classes in the library. For each class, it includes a list of methods, and for each method, it stores two fields: \texttt{methodSignature}, which holds the method’s signature, and \texttt{unchecked\_exceptions}, which lists all unchecked exceptions associated with the method. The tool uses this \texttt{JSON} file to compare the current and latest versions of the library. We remove any duplicate exceptions that appear in both versions. If a method in the latest version throws an exception not present in the older version, we tag that method as having a newly added unchecked exception. Note that our methodology does not detect cases where a method throws the same exception but under new conditions.

% Our tool detects newly added unchecked exceptions by scanning method implementations, and approximates their reachability from the client using taint analysis. After collecting the unchecked exceptions from both the current and the latest versions of the library, we compare the sets of exceptions associated with each method in the library. We remove any duplicate exceptions that appear in both versions. If a method in the latest version throws an exception that was not present in the older version, we tag that method as having a newly added unchecked exception. Note that our methodology does not detect instances where the same exception is thrown by a method, but under new circumstances.

This output enables our tool to highlight call sites in the client application that may exhibit behavioural breaking changes, helping developers assess the impact of upgrading their dependencies.
