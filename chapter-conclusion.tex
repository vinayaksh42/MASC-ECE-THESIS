%======================================================================
\chapter{Conclusion}
%======================================================================
In this work, we demonstrated the impact of behavioural breaking changes caused by newly added unchecked exceptions in client applications. These changes are particularly difficult to detect, as they evade Java's compile-time checks and are not reflected in API signatures.

We introduced UnCheckGuard, a static analysis tool designed to detect such exceptions and help client developers avoid behavioural breaking changes. By combining extracted information with taint analysis, UnCheckGuard filters out unreachable exceptions, focusing only on those that are actually triggerable by client inputs.

We evaluated 99 library–client pairs from the DUETS dataset. Our tool flagged 285 client callsites that invoked library methods with newly added unchecked exceptions. After applying taint analysis, we reduced this to 49 callsites that could potentially trigger an exception at runtime. We wrote manual test cases for these callsites and confirmed that 3 of them resulted in real behavioural breaking changes.

These 49 callsites came from 8 distinct libraries. Of those 8 libraries, 7 introduced newly added unchecked exceptions during a major version upgrade, which aligns with the expectations of semantic versioning. However, one case occurred during a patch-level upgrade (\texttt{httpcore-4.4.6}$\rightarrow$\texttt{httpcore-4.4.16}), indicating that breaking changes can also occur in updates that developers might assume are safe.

UnCheckGuard addresses a concerning gap in existing tools by targeting behavioural breaking changes due to unchecked exceptions. By statically analyzing both the library and client, it provides an effective way to catch runtime issues early and improve software robustness.
