%======================================================================
\chapter{Conclusion}
%======================================================================
In this work, we demonstrated the impact of behavioural breaking changes caused by newly added unchecked exceptions in client applications. These changes are particularly difficult to detect, as they evade Java's compile-time checks and are not reflected in API signatures.

We introduced UnCheckGuard, a static analysis tool designed to detect such exceptions and help client developers avoid behavioural breaking changes. By combining extracted information with taint analysis, UnCheckGuard filters out unreachable exceptions, focusing only on those that are actually triggerable by client inputs.

We evaluated 352 library–client pairs from the DUETS dataset. Our tool flagged 15678 client callsites that invoked external methods. After applying taint analysis, we reduced this to 1708 callsites that could potentially trigger an exception at runtime. We wrote manual test cases for these callsites and confirmed that 3 of them resulted in real behavioural breaking changes.

These 1708 callsites came from 120 distinct libraries. Of those 120 libraries, 50 introduced newly added unchecked exceptions during a major version upgrade, 57 introduced it during a minor version upgrade and the rest during patch version upgrade. While we are not making any claims about how frequently \glspl{bbc} happen in general, but our results indicate that minor and patch upgrades do introduce \glspl{bbc} by addition of unchecked exceptions.

UnCheckGuard addresses a concerning gap in existing tools by targeting behavioural breaking changes due to unchecked exceptions. By statically analyzing both the library and client, it provides an effective way to catch runtime issues early and improve software robustness.
